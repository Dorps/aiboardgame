\documentclass{article}

\usepackage{tabularx}
\usepackage{booktabs}
\usepackage{hyperref}
\usepackage{supertabular}
\usepackage{indentfirst}

\pagestyle {plain}
\pagenumbering {arabic}

\title{Problem Statement and Goals\\\progname}

\author{\authname}

\date{}

\input{../Comments}
%% Common Parts

\newcommand{\progname}{SE 4G06} % PUT YOUR PROGRAM NAME HERE
\newcommand{\authname}{Team \#6, Board Gamers
\\ Ilao Michael, ilaom
\\ Bedi Hargun, bedih
\\ Dang Jeffery, dangj12
\\ Ada Jonah, karaatan
\\ Mai Tianzheng, mait6} % AUTHOR NAMES                  

\usepackage{hyperref}
    \hypersetup{colorlinks=true, linkcolor=blue, citecolor=blue, filecolor=blue,
                urlcolor=blue, unicode=false}
    \urlstyle{same}
                                


\begin{document}

\begin{table}[hp]
\caption{Revision History} \label{TblRevisionHistory}
\begin{tabularx}{\textwidth}{llX}
\toprule
\textbf{Date} & \textbf{Developer(s)} & \textbf{Change}\\
\midrule
Sept $23^{rd}$ & Jonah Ada and Tianzheng Mai & Version 1\\
Sept $24^{th}$ & Michael Ilao & Revised Goals\\
\bottomrule
\end{tabularx}
\end{table}

\newpage

\maketitle



\section{Problem Statement}


%\wss{You should check your problem statement with the
%\href{https://github.com/smiths/capTemplate/blob/main/docs/Checklists/ProbState-Checklist.pdf}
%{problem statement checklist}.}
%\wss{You can change the section headings, as long as you include the required information.}

\subsection{Problem}
\indent Designing a board game is a complicated process.
Balancing a board game is even more complex. And, contrarily to software games
that can support hot patches, ``patching'' a board game after it was printed to
fix a mistake is at best expensive and at worst impossible.
Software patches to the game mechanics and deploying a new release removing the problems can solve the issues for a virtual game but this would have been totally impossible in a
non-virtual environment.


In this project, we have partnered with Bellows Intent, a studio developing a
board game named ``An Age Contrived''
(\url{https://boardgamegeek.com/boardgame/357890/age-contrived}). The game is
part of the ``engine building'' category, like Scythe
(\url{https://boardgamegeek.com/boardgame/169786/scythe}). 

The objective of the project is to use Bellows Intent's game to facilitate the development of an open-source engine that can support running thousands of simulations for their game, helping identify
pitfalls in the game mechanics (e.g., a strategy that always win), as well as
balancing the scoring system. 


We will first focus on Bellows Intent's game to validate that it is
possible to support the design of a game with AI-based techniques. Then we will
assess the reusability of the design gaming framework by applying it to another
game (e.g., Scythe).

\subsection{Inputs and Outputs}

%\wss{Characterize the problem in terms of ``high level'' inputs and outputs.  
%Use abstraction so that you can avoid details.}
Inputs will consist of different variables which the player can manipulate during the game.

Output will consist of the effect of the user input on the overall game state.\\ 
The main purpose is to determine the user input's effect on the game state and find loopholes in the game rule to fix them before publishing the game to public.

\subsection{Stakeholders}
\begin{table}[hp]
\caption{Revision History} \label{TblRevisionHistory}
\begin{tabularx}{\textwidth}{p{5cm}p{5cm}p{1.5cm}}
\toprule
Bellows Intent & Publisher of The Game "An Age Contrived" & Primary\\
\midrule
Chris Matthew & Principal Author of "An Age Contrived" and The Game Designer & Primary\\
\midrule
Dr. Sebastien Mosser & Supervisor & Secondary\\
\midrule
Dr Vladimir Reinharz & Supervisor & Secondary\\
\midrule
Group \# 6 & Developers & Tertiary\\
\midrule
Players & & Tertiary\\
\bottomrule
\end{tabularx}
\end{table}

\noindent 


\subsection{Environment}

%\wss{Hardware and software}
\textbf{Hardware}: Simulation and AI will run on Alliance Canada's (\url{https://alliancecan.ca/en}) computer cluster Canada Compute  to test millions of different possibilities in a reasonable time frame.

\textbf{Software}: will consist of object-oriented modelling to support the representation
of the board game (game engine), command module and player agents modules which will integrate state-of-the-art libraries (such as
\url{https://www.gymlibrary.ml/}, \url{https://pettingzoo.farama.org/}, \url{https://github.com/davidADSP/SIMPLE}) to code artificial intelligent agents able to play thousands of
games.

\section{Goals}
\subsection{Simulation and Learning}
\begin{itemize}
\item Explanation: Implement artificial intelligent players into the system to play thousands of games efficiently to detect patterns and winning strategies.
\item Importance: This is the main focus of the project to test and simulate a board game for quality assurance and balancing before release.
\end{itemize}

\subsection{Efficiency}
\begin{itemize}
\item Explanation: Keep the run time of the simulation and AI game players below a threshold where the system can be easily run on the developer's computers.
\item Importance: This goal will allow all developers to run the simulation on their machines, to get the best data from the system and not have to rent out computing space which would increase costs.
\end{itemize}

\subsection{Complex Problem Solving}
\begin{itemize}
\item Explanation: The system must use AI game players to analyze complex decision paths. Each different AI should have a different decision-making process.
\item Importance: This goal will make sure the output of the system is of quality to the game designers since games will be played intelligently and with different strategies.
\end{itemize}

\subsection*{Quality Data}
\begin{itemize}
  \item Explanation: The system must track and save decisions, game states and patterns throughout the simulations, as well as compare simulations against each other to find winning strategies. The output should be easily readable for non-technical users to understand.
  \item Importance: This goal will make sure the game designers can use the system to balance their game.
\end{itemize}


\section{Stretch Goals}
\subsection{Reusability}
\begin{itemize}
\item Explanation: The system should be highly modular, where different AI Game agents can be swapped in and out to provide different outcomes. The game engineer should also be able
to be swapped in and out. The architecture should allow for a simple game like tic-tac-toe and a complex game like An Age Contrived to fit seamlessly into the system.
\item Importance: This goal improves the quality of the code and should be a side-effect of the architecture. This would make the system more useful as it can test more board games than it was initially designed for.
\end{itemize}

\subsection{User Interface}
\begin{itemize}
\item Explanation: The system should have a live user interface where a user can view a game in real-time, to see what actions the AI game players are making.
\item Importance: This goal will make the system easier to use and allow non-technical users to verify it is functional.
\end{itemize}

\subsection{Availability}
\begin{itemize}
  \item Explanation: The system should be deployed on a server, where the it can run for a longer period of time to gather data.
  \item Importance: This goal will allow us to collect more data which can help further analysis of game balancing.
\end{itemize}
\end{document}
