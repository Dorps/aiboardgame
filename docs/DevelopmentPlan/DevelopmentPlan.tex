\documentclass{article}

\usepackage{booktabs}
\usepackage{tabularx}

\title{Development Plan\\\progname}

\author{\authname}

\date{}

\input{../Comments}
%% Common Parts

\newcommand{\progname}{SE 4G06} % PUT YOUR PROGRAM NAME HERE
\newcommand{\authname}{Team \#6, Board Gamers
\\ Ilao Michael, ilaom
\\ Bedi Hargun, bedih
\\ Dang Jeffery, dangj12
\\ Ada Jonah, karaatan
\\ Mai Tianzheng, mait6} % AUTHOR NAMES                  

\usepackage{hyperref}
    \hypersetup{colorlinks=true, linkcolor=blue, citecolor=blue, filecolor=blue,
                urlcolor=blue, unicode=false}
    \urlstyle{same}
                                


\begin{document}

\begin{table}[hp]
\caption{Revision History} \label{TblRevisionHistory}
\begin{tabularx}{\textwidth}{llX}
\toprule
\textbf{Date} & \textbf{Developer(s)} & \textbf{Change}\\
\midrule
Sept $18^{th}$ & Michael Ilao & Tech Stack, POC, Coding Standard\\
Sept $22^{nd}$ & Hargun Bedi & Team Meeting Plan, Team Communication Plan, Workflow Plan\\
Sept $24^{th}$ & Michael Ilao & Project Schedule and Member Roles\\
\bottomrule
\end{tabularx}
\end{table}

\newpage

\maketitle

This document will outline the development and documentation plan for An AI-based Approach to Designing Board Games project.
\section{Team Meeting Plan}
The team has decided to have weekly meetings on Fridays, from 11:30am to 12:30pm on our Microsoft Teams group. Additionally, the team can have ad-hoc meetings if the team unanimously deems it necessary. 
\\The team has also scheduled weekly checkpoint meetings with our supervisor, Dr. Sebastien Mosser, on Tuesdays, from 5:00pm – 6:00pm. 
\subsection{Meeting Rules}
\begin{itemize}
    \item Team leader will chair the meeting.
    \item All team members will give an update on their assigned tasks.
    \item All agenda items will be discussed sequentially.
    \item All team members will have an opportunity to ask questions and voice any concerns they have.
    \item If a new task is being assigned to any member, the team should decide the appropriate timeline unanimously. 
    \item If a member requests a revision to the meeting time, agenda, or deadline, they must provide a 48-hour notice and the changes will be agreed upon unanimously.
    \item At the end of each meeting, an agenda will be made for the next meeting. 
\end{itemize}

\section{Team Communication Plan}
The team’s communication will be done virtually, primarily on our Microsoft Team’s group chat. Any questions, concerns, requests to change deadlines, or general discussions will be done on the Team’s group chat. If necessary, we will also be meeting in-person, at the team’s decided location prior to the meeting. 
Additionally, all emails sent to our supervisors, professor, or TA’s must have all team members CC’d on it.
\section{Team Member Roles}
As the architecture will require two main parts of development, our developers effort will be focused so they can understand their specific system on a deeper level.
The two roles will be AI Game Agent Developer, who will work on the Artificial Intelligence (AI) Game Agent. The Simulation Structure, developers will focus on the architecture of the game engine and how it connects to the AI Game Agent
\begin{itemize}
	\item Jonah: AI Game Agent -\ Developer
	\item Jeffery: AI Game Agent -\ Developer
	\item Hargun: Simulation Agent -\ Developer
	\item Michael: Simulation Structure -\ Developer
	\item Tianzheng: Simulation Structure -\ Developer
\end{itemize}
All the developers will be required to understand the system at a high-level on both parts. They will also be required to work on documentation, testing, requirements and attend and be active in all meetings with the Stakeholders.
\section{Workflow Plan}
Teammates are required to use the private GitHub repository, AIBoardGame. 
In order to facilitate simultaneous development of modules, we will be using the “feature-branch” workflow. First, each team member will pull the latest changes from the “main” branch into their local repository. 
Next, changes will be made by creating a new “feature” branch. It is expected that feature branches will be given descriptive names, like "ai-implementation" or "issue \#31". The idea is to give a clear purpose to each branch. 
Once the changes are made, unit testing will be performed on the feature branch to ensure the code is working as intended. 
When the feature branch is merged with the main branch, 1 or 2 other team members may do integration testing to verify the if the merge did not cause any other issues. 

\subsection{Issue Tracking}
All tasks will be completed and tracked using 4 label categories on GitHub issue tracker – \emph{Major, Minor, Bug,} and \emph{New Feature}. Each issue description should include all necessary details to completely understand the issue, and will be using one or more of the labels mentioned below. Issues can be assigned to the team member that can best solve the issue.

\emph{Major} label will be used when the main branch is not able to run or if the code keeps crashing on a specific action.

\emph{Minor} label will be used when there are issues identified with the UI, inadequate error handling, uncaught errors in unit/integration testing, etc.

\emph{Bug} label will be used when there are unintended results due to logical errors, inputs not being handled properly, outputs not being generated properly, etc.

\emph{New Feature} label will be used when a new feature or improvement is being added to the project.
\\All issues will be required to be fixed in a separate branch with an appropriate issue name. All Issues will be reported to the team either in the team meeting or in the team group chat. Issues will be closed when the appropriate pull request is merged to the main branch by including a closing branch name/description (close issue \#31).

\subsection{Tags}
Tags will be used to mark milestones in the development of the project. Major milestones will be incrementing the major version number (v2.0 to v3.0). Minor milestones will be incrementing the minor version number (v2.1 to v2.2). The team will decide on an ad-hoc basis on what will be considered a major/minor milestone as the project is in its development stage.

\section{Proof of Concept Demonstration Plan}

For a proof of concept demonstration, the base architecture must be working and implemented properly. This means that the simulation and AI Game Agent are fully
module and can be swapped with different AI Agents or even a different simulation engine. The simulation engine to be implemented will be a simple game of tic-tac-toe.
If the integration of AI Game Agents can play on the simulation of tic-tac-toe, the project has a great chance of success as we will know the architecture is able to support
more complex game agents and a more complex simulation engine. These two systems will be developed in parallel to maximize our time constraints.
The major risk of this project is how to create a modular architecture and able to integrate a simulation engine that can be played by real players as well as AI players. Testing this will not be difficult for the POC as the game chosen Tic-Tac-Toe is a simple game and we can test the success of our AI's quickly as many simulations can be ran quickly.

In our demonstration we will be able to show that are architecture is modular enough to support different kinds of AI Game Agents and that they are seamlessly able to integrate into the simulation (By swapping different AI Game Agents in/out).
\section{Technology}

\begin{itemize}
\item Python will be used to develop the simulation engine and be used for simulation the AI players. 
The choice of this language is due to Python's Machine Learning/Artificial Intelligence libraries and the support for
 Object Oriented Programming.
\item Object Oriented Programming will be used as the design methodology as to allow multiple developer to work seamlessly 
on the same project and for extensibility to other possible board games/simulations. 
\item To ensure common programming standards, developers will use pylint to maintain the same coding style across files. 
VSCode and prettier will be used for automatic formatting and linting.
\item Pytest will be used for integration and unit testing for the Simulation Engine. For testing the AI Game engine, developers will need to build custom test to verify they are working to an acceptable degree.
\item Coverage.py will be used for code coverage as it integrates easily with pytest.
\item There are no immediate plans for Continuous Integration/Continuous Deployment as the project will be used for testing by the 
Stakeholders, which does not need to be hosted on any cloud environment.
\item timeit Python library will be used for measuring performance and time of individual modules.
\item ML/AI Libraries to be used will be PyTorch, Tensor Flow, NumPy and Pandas.
\item The main tools used will be VSCode and any available Python extensions.
\end{itemize}

\section{Coding Standard}
The programming paradigm that will be used in this project is Object Oriented Programming or OOP.This will allow to developers to structure code
for re-use and extensibility. This abstract way of programming will allow for the system to be integrate into different board games. Another standard that
will be used is PascalCase for Class naming and camelCase for method and variable naming. PEP8 will also be used to enforce readable code and good python practices.

\section{Project Scheduling}
The project schedule will require developers to work on documentation and planning in parallel with the code. This will benefit developers as they will be able to adjust their 
methodologies and documentation faster than if it was done sequentially.

The project will be managed with GitHub Projects by assigning tasks to team members.

The Major milestones are the Requirements Doc Rev 0, POC Demo, Rev 0 Demo and Final Demo. These tasks will be broken down and split up by different sections that each team member can individually contribute to.
Documentation will be assigned to team members based on what their dev-role is, AI Game Agent Developers will work on documentation related to this part of development and
Simulation Developers will work on documentation relating to that. This will mean team members may be working on the same section together. At a high level all team members must still understand each part of the project and system.

Communication between the team will be a major factor to fully utilize each team members time. As there are multiple people working on the same part, they must work collaboratively to ensure there is no code or documentation overlap. 
\end{document}
